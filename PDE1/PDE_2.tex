\documentclass[11pt]{amsart}% use "amsart" instead of "article" for AMSLaTeX format
\usepackage{geometry}                		% See geometry.pdf to learn the layout options. There are lots.
%\geometry{letterpaper}                   		% ... or a4paper or a5paper or ... 
%\geometry{landscape}                		% Activate for for rotated page geometry
%\usepackage[parfill]{parskip}    		% Activate to begin paragraphs with an empty line rather than an indent
\usepackage{graphicx}				% Use pdf, png, jpg, or eps§ with pdflatex; use eps in DVI mode
								% TeX will automatically convert eps --> pdf in pdflatex	
\usepackage{amsmath}
\usepackage{mathtools}
\usepackage{graphicx}
\usepackage{amssymb}
\usepackage{esint}
\usepackage{enumerate}
\usepackage{epstopdf}
\newtheorem{theorem}{Theorem}
\newtheorem{proposition}[theorem]{Proposition}
\newtheorem{lemma}[theorem]{Lemma}
\newtheorem{definition}[theorem]{Definition}
\newtheorem*{claim}{Claim}
\newtheorem{example}{Example}
\newtheorem{remark}{Remark}
\newtheorem{corollary}{Corollary}
\DeclareGraphicsRule{.tif}{png}{.png}{`convert #1 `dirname #1`/`basename #1 .tif`.png}
\newcommand\F{\mathbb{F}} \newcommand\E{\mathbb{E}} \newcommand\R{\mathbb{R}}
\newcommand\Z{\mathbb{Z}} \newcommand\N{\mathbb{N}} \newcommand\C{\mathbb{C}}
\newcommand\Q{\mathbb{Q}} \newcommand\eps{\varepsilon}
\newcommand\cM{\mathcal{M}}	\newcommand\cA{\mathcal{A}}
\newcommand\cC{\mathcal{C}} \newcommand\cB{\mathcal{B}}
\newcommand\cE{\mathcal{E}}
\DeclareMathOperator{\sech}{sech}
\DeclareMathOperator{\dt}{\frac{d}{dt}}
\DeclareMathOperator{\pt}{\frac{\partial}{\partial t}}
\DeclareMathOperator{\dive}{div}
\DeclareMathOperator{\grad}{grad}



\title{Partial Differential Equations 1}
\author{Son Van}
\date{Spring 2016}

\begin{document}
\maketitle
This is part 2 of the notes.

\begin{center}\rule{0.5\linewidth}{\linethickness}\end{center}

\textbf{Monday, Feb 15 2016}

\begin{center}\rule{0.5\linewidth}{\linethickness}\end{center}

Today we will talk about another approach to the BVP using potential theory.

\[\begin{cases}
    -\Delta u=f \quad \text{ in }U\\
    u=g \quad \text{ on } \partial U
\end{cases}\]
We have
$$u(x)=\int_U \Phi(y-x)(-\Delta u)(y) dy + \int_{\partial U} \Phi(y-x)\frac{\partial u}{\partial \nu} -u(y)\frac{\partial \Phi}{\partial \nu}(y-x) dS(y)$$


The goal is we want to get rid of 
$$\Phi(y-x)\frac{\partial u}{\partial \nu}$$
and replace $u(y)$ by some $\phi(y)$ which is termed ``double layer potential".

We want to develop a boundary integral equation to solve for $\phi$ so that
$$u(x)\to g(x_0), x\to x_0 \in \partial U$$
and $$g(x)= \Phi * f(x_0) \pm \frac{1}{2}\phi(x_0) -\int_{\partial U} \phi(y)\frac{\partial \Phi}{\partial \nu} (y-x_0)dS(y).$$

We look at the Direchlet's principle (variational method). Let $$A_g=\{ v\in C^2(U)\cap C^1(\overline{U})| v= g \text{ on } \partial U\}$$
$$A_0 = \{ v\in C^2(U)\cap C^1(\overline{U})| v=0 \text{ on } \partial U\}.$$

1. Suppose $f,g$ are continuous and $u\in A_g$ satisfies BVP, then for all $v\in A_0$,
$$\int_U fv = \int_U f(-\Delta u) dx = \int_U Dv\cdot Dv + \int_{\partial U} v\frac{\partial u}{\partial \nu}$$ where $v\frac{\partial u}{\partial \nu}=0$ on the boundary. 

In other words, $u$ satisfies
$$\int_U Du\cdot Dv dx =\int_U fv dx, \forall v\in A_0.$$
This is called the weak form of BVP.

2. Define, for any $u\in A_g$ the functional
$$I[u] = \int_U(\frac{1}{2}vert Du\vert^2 -fu) dx$$
$$I: A_g \to \R.$$

\begin{theorem}
    Assume $U$ is open bounded on $\R^n$, $\partial U\in C^1$. Assume $f,g$ are continuous, $u\in A_g$. Then $u$ satisfies BVP iff
    \[ I[u]=\min_{v\in A_g} I[v]. \tag{*}\]
\end{theorem}

\begin{proof}
    $(\Leftarrow)$. Suppose (*) holds, fix $v\in A_g$ arbitrary. Define $l(t) = I[u+tv]$ ($u+tv \in A_g$), $l:\R\to\R$ and $l(0)=\min l$ and $l$ is quadratic. So,
    \begin{eqnarray*}
        0&=&l'(0) = \frac{d}{dt}\int_U \frac{1}{2} \vert Du + tDv\vert^2 -f(u+tv) \vert_{t=0}\\
        &=& \int_U Du\cdot Dv -fv, \forall v\in A_0.
    \end{eqnarray*}
    Integrate by parts,
    $$\int_U Du\cdot Dv = \int_U(-\Delta u)v +\int_{\partial U} \frac{\partial u}{\partial \nu} v.$$
    So, $0 =\int_U(-\Delta u-f)v$ for all $v\in A_0$. Thus,
    $$\begin{cases}
        -\Delta u =f \quad \text{ in } U\\
        u=g \quad \text{ on } \partial U
    \end{cases}.$$
    
    $(\implies)$. Suppose $u\in A_g$ and satisfies BVP. Let $w\in A_g$. We claim $I[w]\ge I[u]$. Then $v=w-u\in A_0$ and $w=u+v$.
    \begin{eqnarray*}
        I[u+v] &=& \int_U\frac{1}{2}\vert u+v\vert^2 -f(u+v)\\
            &=& I[u] + \int_U Du\cdot Dv -fv +\int_U\frac{1}{2}\vert Dv\vert^2 \ge I[u]
    \end{eqnarray*}
\end{proof}

\begin{center}\rule{0.5\linewidth}{\linethickness}\end{center}

\section*{The heat (diffusion) equation}

$$u_t =\Delta u.$$

Origin: flow by curvature. Consider a curve in $\R^2$ moving parametrized by $(x,t)\mapsto \overline{u}(x,t)\in \R^2$, $x\in [0,1], t\in \R$. Then the unit tangent at time $t$ is
$$\tau = \frac{u_x}{\vert u_x\vert}.$$
Element of arclength $ds = \vert u_x\vert dx$.

Curvature: $$\kappa =\frac{d\tau}{ds}= \frac{u_{xx}}{\vert u_{xx}\vert^2} -\frac{u_{x}}{\vert u_x\vert^3}(\frac{u_x}{\vert u_x\vert} \cdot u_{xx}).$$

We seek to define velocity $u_t$ so that normal velocity = curvature.
$$u_t = \frac{\kappa}{\vert \kappa\vert} = \vert \kappa\vert$$ since $\kappa\cdot \tau=0$. It suffices to set $$u_t =\frac{u_{xx}}{\vert u_x\vert^2} (= \kappa +c(t)\cdot \tau).$$ This is a nonlinear version of the heat equation.

\begin{center}\rule{0.5\linewidth}{\linethickness}\end{center}

\textbf{Wednesday, Feb 17 2016}

\begin{center}\rule{0.5\linewidth}{\linethickness}\end{center}

Denote $R_+=(0,\infty)$.

We want to hunt for the fundamental solution. The IVP is given $g:\R^n\to \R$. Find $u:\R^n\times [0,\infty)\to \R$ so that
$$\begin{cases}
    u_t=\Delta u, \quad x\in \R^n, t>0\\
    u(x,0) =g(x)
\end{cases}$$  

\begin{definition}
    We say $u\in C^{2,1}$ on some set if $u, u_t, D_xu, D_x^2u$ exist and are continuous on the set.
\end{definition}

Symmetry for $u_t=\Delta u$ in $\R^n\times \R_+$. Suppose $u\in C^{2,1}(\R^n,\R_+)$ is a solution. Then $v(x,t)$ is also in each case:

\begin{enumerate}
    \item (Translate) $\forall y\in \R^n$, $v(x,t)=u(x-y,t)$
    \item (Rotation) For any $n\times n$ orthogonal matrix $Q$, $v(x,t)=u(Qx,t)$
    \item (Scaling) $\forall \lambda >0, v(x,t)=u(\lambda x, \lambda^2 t)$
    \item (Superposition) $\forall g\in C_c(\R^n)$, $u(x,t) =\int_{\R^n} u(x-y,t) g(y) dy$
\end{enumerate}

{\emph Heuristic.} To solve IVP, represent
$$g(x)=\int_{\R^n}\delta(x-y)g(y)dy$$
 we seek to find $u(x,t)$ so that
 \begin{eqnarray*}
    ``u(x,t) &\to& \delta(x)''\\
    t   &\to& 0.
 \end{eqnarray*}
 We expect $u(x,t)$ to have one symmetry as $\delta(x)$. So, we look for
 \begin{enumerate}
     \item $u(Qx,t)=u(x,t)$ for all orthogonal $Q$
     \item $u(x,t) = A(\lambda)u(\lambda x, \lambda^2 t)$ where $A(\lambda)=\lambda^n$ to make sure
     $$1=\int_{\R^n}\delta(x)dx = \lim_{t\to 0}\int_{\R^n} u(x,y) dx =\int u(\lambda x, \lambda^2 t)\lambda^n dx.$$
 \end{enumerate}
 
 Suppose $u$ has properties (1), (2), then given $(x_0, t_0)$ we choose $Q$ so $Qx_0=\vert x_0\vert e_1$. Choose $\lambda_0 =\frac{1}{\sqrt{t_0}}$. Then $u(x_0, t_0) =u(\vert x_0\vert e_1, t_0)=t^{-n/2}_0 u(\frac{\vert x_0\vert e_1}{\sqrt{t_0}},1)$. Thus, we expect
 $$u(x,t)=t^{-n/2} v(\frac{\vert x\vert}{\sqrt{t}}).$$ We could replace $\frac{\vert x\vert}{\sqrt{t}}$ by $\frac{\vert x\vert^2}{{t}}$. So, $$u(x,t)=t^{-n/2} v(\frac{\vert x\vert^2}{t}).$$
 $$v:[0,\infty)\to \R.$$
 
 ODE for $r$: let $r=\vert x\vert, z=\frac{r^2}{t}$, then
 \begin{eqnarray*}
    u_t &=& -\frac{n}{2} t^{-\frac{n}{2} -1}v(z) + t^{-n/2}v'(z)(-\frac{r^2}{t^2})\\
        &=& (t^{-\frac{n}{2}-1})(-\frac{n}{2}v(z) - zv'(z))
 \end{eqnarray*}
 
 \begin{eqnarray*}
    \Delta u &=& (\partial_r^2 + \frac{n-1}{r}\partial_r)(t^{-n/2}v(z)) = t^{-\frac{n}{2}-1} [4zv''(z) + 2nv'(z)]
 \end{eqnarray*}
 
 So, $u_t =\Delta u$ if
 \begin{eqnarray*}
    4zv''(z) + (2n+z)v'(z) + \frac{n}{2} v(z) =0\\
    \iff z(4v''+v') + \frac{n}{2}(4v' + v) = 0.
 \end{eqnarray*}
 We may choose $$v=e^{-z/4} a$$ so $$u(x,t) = \frac{a}{t^{n/2}} e^{-\vert x^2\vert}{4t}.$$
 
 Define: $y=\frac{x}{2\sqrt{t}}$.
 $$1=\int u(x,t)dx = 2^n a\int e^{-\vert y\vert^2} dy = 2^n\prod_{j=1}^n(\int_\R e^{-y^2_j}dy_j).$$
 
 Let $a=(4\pi)^{-n/2}$, then
 $$u=\frac{1}{4\pi t)^{n/2}} e^{-\vert x\vert^2/4t} =: \Phi(x,t).$$
 $\Phi$ is the fundamental solution of $u_t=\Delta u$ in $\R^n$.
 
 {\bf Properties.}
 \begin{enumerate}
     \item $\Phi$ is $C^\infty$ on $\R^n\times \R_+$
     \item $\partial_t \Phi =\Delta \Phi$
     \item $\int_{\R^n} \Phi(x,t)dx =1$ for all $t>0$
     \item $\Phi>0$ and for all $\delta>0$, $\sup_{\vert x\vert >\delta} \Phi(x,t)\to 0$ as $t\to 0$.
     $$u(x,t)=\int_{\R} \Phi(x-y,t)g(y) dy.$$
 \end{enumerate}
 
 For last class, motion by curvature shortening. Suppose $$(x,t)\mapsto u(x,t)$$ is smooth and 1-periodic in $X$. Assume $\vert u\vert \not=0$. Let $\Gamma(t)$ be the image of $x\mapsto u(x,t)$. The arclength
 $$\vert \Gamma(t)\vert =\int ds =\int_{[0,1]}\vert u_x\vert dx.$$
 
 Claim. The following is true
 $$\frac{d}{dt}\vert \Gamma(t)\vert = -\int \kappa\cdot u_t dx.$$
 
 Why: \begin{eqnarray*}
    LHS &=& \int_{[0,1]} \frac{u_x}{\vert u_x\vert} u_{xt} dx = \int_{[0,1]} \frac{u_x}{\vert u_x\vert} (u_t)_x dx\\
    &=& \int -\frac{\partial}{\partial x} (\frac{u_x}{\vert u_x\vert})\cdot u_t dx + BC\\
    &=& -\int \frac{\partial}{\partial s}(\tau)\cdot u_t ds=-\int \kappa\cdot u_t ds
 \end{eqnarray*}
 where $\tau=\frac{u_x}{\vert u_x\vert}$. 
 
 If $z:\R\to \R^n$, $F:\R^n\to \R$,
 $$\frac{d}{dt}(F(z(t)) = \nabla_z F\cdot z'.$$
 ``$\kappa = -\grad\vert \Gamma(t)\vert$'' in functional sense.
 
 \begin{center}\rule{0.5\linewidth}{\linethickness}\end{center}

\textbf{Friday, Feb 19 2016}

\begin{center}\rule{0.5\linewidth}{\linethickness}\end{center}

{\bf Weak maximum principle.} Assume $U\subseteq \R^n$ is open and bounded. For $T>0$, put $$U_T=U=(0,T],$$ $$\Gamma_T=\partial U\times [0,T] \bigcup U\times\{0\} =\partial U_T\backslash(U\times\{T\}).$$

\begin{theorem}
    Assume $u\in C^{2,1}(U_T)\cap C^1(\overline{U_T})$ and
    $$u_t -\Delta u \le 0$$ in $U_T$. Then, for all $(x,t) \in \overline{U_T}$, $u(x,t)\le \max_{\Gamma_T} u$, i.e.,
    $$\max_{\overline{U_T}} = \max_{\Gamma_T} u.$$
\end{theorem}

\begin{proof}[''Proof"]
    If $(x,t)\in U_T$ is a max, then at $(x,t)$,
    $$u_t\ge 0, Du=0, \frac{\partial^2 u}{\partial x_i^2} \le 0$$ so $$u_t-\Delta u \ge 0.$$
\end{proof}

\begin{proof}
    For $\epsilon >0$, let $u^\epsilon(x,t)=u(x,t)-\epsilon t$. Then $u_t^\epsilon - \Delta u^\epsilon \le 0-\epsilon <$ in $U_T$.
    
    We claim that $u^\epsilon$ has no max in $U_T$. To see this, if $(x,t)$ is a max, then at $(x,t)$
    $$u_t^\epsilon \ge 0, D_xu^\epsilon =0, \frac{\partial^2 u^\epsilon}{\partial x_i^2}\le 0.$$ This means
    $$u_t^\epsilon -\Delta u^\epsilon \ge 0$$ which is a contradiction. Thus,
    $$\max_{\overline{U_T}}u^\epsilon = \max_{\Gamma_T} u^\epsilon.$$ At any $(x,t)\in \overline{U_T}$,
    $$\vert u(x,t) - u^\epsilon(x,t)\vert \le \epsilon T.$$
    Assume as $\epsilon\to 0$, $u^\epsilon - u \to 0$ uniformly in $\overline{U_T}$. 
    
    One may deduce, as $\epsilon\to 0$
    $$\max_{\overline{U_T}} u^\epsilon\to \max u$$
    and thus
    $$\max_{\Gamma_T} u^\epsilon \to \max_{\Gamma_T} u.$$
\end{proof}

{\bf Energy method.} Uniqueness and stability.

\begin{theorem}
    Assume $U$ is bounded and open and $\partial U$ is $C^1$. Suppose $f:U\to \R, g: \Gamma_T \to \R$ are continuous. Consider the IBVP
    $$\begin{cases}
        u_t-\Delta u = f \qquad, U_T\\
        u=g \qquad, \Gamma_T
    \end{cases}.$$
    Then there can be at most 1 solution $u\in C^{2,1}(U_T)\cap C^1(\overline{U_T})$.
\end{theorem}

\begin{proof}
    Suppose $u, \tilde{u}$ are 2 solutions. Let $w=u-\tilde{u}$, we have
    $$\begin{cases}
        w_t -\Delta w =0 \qquad, U_T\\
        w=0 \qquad, \Gamma_t
    \end{cases}.$$
    
    Multiply by $w$ and integrate in $X$,
    $$\int_U w\frac{\partial w}{\partial t} dx =\int_U w\Delta w dx$$
    
    $$\frac{d}{dt}\int_U \frac{w^2}{2}dx = -\int_U \vert Dw\vert^2 dx \le 0.$$
    
    Integrate in $t$,
    $$0\le \int_U w(x,t)^2 dx \le \int_U w(x,0)^2dx=0.$$
    
    Thus, $w(x,t)=0$ in $U$ and $\forall t\in [0,1]$.
\end{proof}

{\bf A stability estimate.} Consider the IBVP
$$\begin{cases}
    u_t -\Delta u =0 \qquad, U_T\\
    u=0 \qquad, \partial U \times [0,T]\\
    u=g \qquad, U\times\{0\}
\end{cases}.$$
Suppose $u\in C^{2,1}(U_T)\cap C^1(\overline{U_T})$, then
$$\int_U\frac{1}{2} u(x,t)^2 dx + \int_0^t\int_U \vert D_x u(x,s)\vert^2 dxds =\int_\Omega \frac{1}{2}g^2(x)dx$$
So, $$\Vert u\Vert_{L^2} \le \Vert g\Vert_{L^2}$$ (changes in $g$ dominate changes in $u$ in $L^2$ sense.)

The ``backwards'' initial value problem: unique but highly unstable.
$$\begin{cases}
    u_t = \Delta u \qquad, x\in U, t\in [0,T]\\
    u=0 \qquad, x\in \partial U, t\in [0,T]\\
    u=g \qquad, x\in U, t=T
\end{cases}$$

\begin{theorem}[Backward uniqueness, see Evans]
    The problem has at most 1 solution, $u\in C^{2,1}(U_T)$.
\end{theorem}

{\bf Ill-posedness.} Take $n=1$, $U=(0,\pi)$. Consider
$$\begin{cases}
    u_t = u_{xx} \qquad, 0<x<\pi, 0<t<T\\
    u=0 \qquad, x=0,\pi
\end{cases}$$

There are solutions
$$u_k(x,t)=e^{-k^2 t} \sin kx, \quad k\in \Z.$$

Put $g_k(x) = e^{-kT}\sin kx$ and look at $\Vert u_k\Vert_{L^2}/\Vert g_k\Vert_{L^2}$.
$$(\int_0^\pi e^{-2k^2 t}\sin^2 kx dx)^{1/2} / (\int_0^\pi e^{-2k^2 T}\sin^2 kx dx)^{1/2}=e^{-k^2(T-t)}.$$

The ratio ``solution/data'' could be arbitrarily large.
\end{document}